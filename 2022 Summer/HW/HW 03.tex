\documentclass{exam}


%==================================================================================
% I. Preambulo
%==================================================================================
%----------------------------------------------------------------------------------
% 1.Paquetes
%----------------------------------------------------------------------------------
\usepackage[spanish]{babel}			% Permite escribir en espanol
\usepackage[utf8]{inputenc}
\usepackage{amsmath}						% Para ecuaciones y demases
\usepackage{amsthm}							% Para ecuaciones y demases
\usepackage{amssymb}						% Para ecuaciones y demases
\usepackage{graphicx} 					% Para insertar graficos
\usepackage{float}							% Para manejar la ubicacion de graficos
\usepackage{verbatim}						% Para escribir codigos
\usepackage{url}								% Para escribir direcciones web
\usepackage{subfig}							% Para poner varias figuras en el mismo marco
\usepackage{psfrag}							% Para hacer reemplazos en las figuras
\usepackage{multicol}
\usepackage{multirow}
\usepackage{bigstrut}
\usepackage{enumerate}
\usepackage{booktabs}
\usepackage{color}
\usepackage{amsfonts}
\usepackage{geometry}

\usepackage{color}
	\definecolor{ceruleanblue}{rgb}{0.16, 0.32, 0.75}
	\definecolor{coolblack}{rgb}{0.0, 0.18, 0.39}
	\definecolor{darkgreen}{rgb}{0.0, 0.2, 0.13}
\usepackage{multirow,hhline}
%\usepackage[linkcolor=blue,colorlinks=true]{hyperref}
\usepackage{tikz}


\spanishdecimal{.}

\usepackage{enumerate}
\usepackage{enumitem}


\usepackage{booktabs}
\usepackage{color}
\usepackage{amsfonts}
\usepackage{geometry}
\usepackage{graphicx}
\usepackage{MnSymbol}

\usepackage{pgf}
\usepackage{etex}
\usepackage{tikz,pgfplots}
 \usepackage[usenames,dvipsnames,svgnames,table]{xcolor} 
 
\usetikzlibrary{matrix,arrows,decorations.pathmorphing}

\usetikzlibrary{calc,matrix}


\tikzstyle{every picture}+=[remember picture]
% By default all math in TikZ nodes are set in inline mode. Change this to
% displaystyle so that we don't get small fractions.
\everymath{\displaystyle}

\usepackage{hyperref}
\hypersetup{colorlinks=true,
urlcolor=blue, linkcolor=blue}

\usepackage{hyperref}
\usepackage[hidelinks]{hyperref}
\hypersetup{pdfborder=0 0 0}

\usepackage{booktabs,caption}
\captionsetup[table]{name=Tabla} 
\usepackage[flushleft]{threeparttable} 


%--------------------------------------

%----------------------------------------------------------------------------------
% 2.Estilo de la pagina
%----------------------------------------------------------------------------------
\pagestyle{headandfoot}					% Opcion para tener headers y footers
\headrule 											% Linea horizontal bajo el header

\pagestyle{headandfoot}          % Opcion para tener headers y footers
\headrule                       % Linea horizontal bajo el header
\firstpageheader{\includegraphics[scale=0.07]{escudo1.pdf} }{}{Universidad de Chile\\
Facultad de Economía y Negocios}
\runningheader{\scriptsize{Universidad de Chile} \\ \scriptsize{Facultad de Economía y Negocios}}{\scriptsize}{\scriptsize{\includegraphics[scale=0.07]{escudo1.pdf}}}
\footrule
\footer{}{\scriptsize{P\'agina \thepage\ de \numpages}}{}
\parindent = 0pt
\renewcommand\partlabel{(\thepartno.)}
\renewcommand\thesubpart{\roman{subpart}}
\footrule
\footer{}{\scriptsize{P\'agina \thepage\ de \numpages}}{}
\parindent = 0pt
\renewcommand\partlabel{(\thepartno.)}
\renewcommand\thesubpart{\roman{subpart}}

%----------------------------------------------------------------------------------
% 3.Formato respuesta
%----------------------------------------------------------------------------------


\noprintanswers          % Esta opcion sirve para esconder las respuestas si
                                % no se quiere imprimir las respuestas, colocar
                                % \printanswers


\renewcommand{\solutiontitle}{\noindent\textbf{Respuesta:}\par\noindent}


%==================================================================================
% II. Documento
%==================================================================================
\begin{document}
\begin{center}
\LARGE{{\textsc{Juegos \& Estrategias}}}\\
\Large{{\textsc{Tarea 3}}}
\end{center}

\hline{}{}
 \begin{flushleft}

\end{flushleft}
\hline{}{}
%--------------------------------------------------------------------------------------------------------------------------------------------------------------------
%Aqui comienza


\section{Problema mundo laboral}

En el mercado laboral hay un gran número de firmas neutrales al riesgo que compiten por trabajadores. Estos pueden ser de dos tipos: de alta ó baja habilidad, los cuales tienen una productividad de \$100 y \$50 cuando no poseen escolaridad. Previo a ingresar al mundo laboral, los trabajadores pueden escoger años de escolaridad. La productividad laboral se incrementa en \$1 por cada año adicional de educación. Acumular un año más de escolaridad tienen un costo (en tiempo y energía) de \$4 para trabajadores de alta habilidad y \$6 para los de baja habilidad. Se pide lo siguiente:

\begin{enumerate}
    
\item Explicite las funciones de utilidad de ambos tipos de trabajadores.
 \subsection*{Solución:}
 
Como hay un gran número de firmas neutrales al riesgo que compiten por trabajadores podemos asumir que el mercado es perfectamente competitivo, es decir, el salario va a ser igual a la productividad
marginal. También asumimos información completa y perfecta, sólo para este item. La función de utilidad para los trabajadores de alta productividad es:

\begin{equation}
    U_alta=100+educ−4educ=100−3educ
    \label{fx utilidad para los trabajadores de alta productividad}
\end{equation}


\item Asuma ahora un contexto de información incompleta, donde los trabajadores conocen privadamente su habilidad. Obtenga el Equilibrio separador. Explique conceptualmente su resultado.

\item Sea $\alpha$ la probabilidad (o frecuencia) de que un trabajador sea de alta habilidad. Proponga un mezclador
asumiendo $\alpha= 0,5$.
\end{enumerate}


\section{Principal Activo: Parte I}

El principal activo de un individuo cuyo nivel de riqueza es $W = 100$, es su casa. Enfrenta un riesgo de incendio con probabilidad $25\%$ , en el que perderá $64$. La utilidad del dinero está dada por la función de utilidad $v(w)$ que cumple la regla de la utilidad esperada, y que tiene la forma $v(w) = w^{1/2} $.

\begin{enumerate}

\item  ¿Cuánto es lo máximo que está dispuesto a pagar por un seguro de cobertura completa?

\item ¿Cuál es la mayor prima (precio) que puede cobrar la única compañía de seguros de la ciudad?

\item Suponga que en la ciudad hay un gran número de individuos exactamente iguales, todos los cuales se aseguran y no hay otros interesados en asegurarse. Suponiendo que el único costo de la única empresa aseguradora es pagar a los asegurados que sufren siniestros, determine los beneficios de la empresa (recuerde la ley de los grandes números ).
\end{enumerate}

\section{Principal Activo: Parte II}

Asuma el mismo escenario que en el apartado anterior. Responda las siguientes preguntas:
\begin{enumerate}
    \item Suponga que una vez comprado el seguro los individuos dejan de tomar precauciones. El cambio de conducta que equivale a un aumento de la riqueza de $\$2$, pero aumenta la probabilidad de que ocurra un incendio $35 \%$. ¿Qué sucede con el mercado?

\item ¿Qué ocurre si la compañía de seguros coloca un deducible de $\$7.5$ y reduce el costo del seguro a $\$17$

\item Muestre que la aseguradora determina la combinación óptima de prima total R y deducible D resolviendo el siguiente problema:

$$ \max_{R,D} R - 0,25 \cdot (64 - D)$$

Sujeto a:

$$0,25 \cdot (100 - R - D)^{1/2} + 0,75 \cdot (100 - R)^{1/2} \geq 9$$

$$0,35 \cdot (102 - R - D)^{1/2} + 0,65 \cdot (102 - R)^{1/2} \leq 9$$

\item Imagine que instalando un sistema de detección de humo que cuesta \$0,2 la probabilidad de incendio se
reduce a 25\% aunque la persona se comporte en forma descuidada. ¿Qué puede ocurrir en el mercado?
\end{enumerate}



\section{Bonus}

Suponga que en el mercado laboral existe un gran número de trabajadores, cuyas productividades laborales individuales a se obtienen de una función de densidad uniforme en el intervalo $[1, 5]$. La productividad fuera del mercado laboral de los trabajadores de tipo a es $r(a) = a - 1$. Asuma también que las firmas son neutras al riesgo y compiten entre ellas en la contratación de trabajadores ofreciendo salarios (w).

\begin{enumerate}
    
\item  Encuentre el equilibrio competitivo bajo el supuesto que a es de público conocimiento, esto es, a es observable para todos los participantes del mercado laboral. ¿Es este equilibrio eficiente? Explique.

\item Encuentre el equilibrio competitivo cuando los trabajadores conocen privadamente sus propias productividades laborales (a). ¿Es este equilibrio eficiente? Explique.

\end{enumerate}
  \end{document}
